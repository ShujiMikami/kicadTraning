\documentclass{jarticle}

\title{1stオーダーボードに対するレポート}
\date{2018/11/17}

\begin{document}
\maketitle
\date

\section{はじめに}
今回初めての基板作成に当たって, 製造の結果や, 納期, 仕上がったものの問題点などまとめておく

\section{製造したもの}
ごく単純な, トランス, ダイオード型の直流発生回路. AC100Vから, DC5V, 3.3Vを生成する. 

\section{製造納期}
今回, Elecrowに発注をかけたところ, 2日でIn Productionにステータスが変わり, 6日後にShippedに変化した. そこからANAを選んだ結果, 2日ほどで到着した. 
つまり, 概ね10日が納期となる. 

\section{製造コスト}
6枚で18ドルなので, おおよそ300円. サイズは40*100なので, 面積に比例して大きくなるかもしれない. ちなみに, 送料が含め, 3200円.

\section{仕上がり}
設計した通りにできていた. ちまたで言われているような, シルクのズレも特にない. 今回単純な基板だったからかもしれない. 

\section{問題点}
そもそも回路図や, フットプリントに間違いがあったのと, パターン設計的によくない部分があった.

\subsection{回路図間違い}
レギュレータの回路で, In側にセラコン, Out側に電解コンをつけなければならなかったのに, それが逆転していた. 

\subsection{フットプリント間違い}
レギュレータは, In,GND,Outの順で, ピン番号としては, 1,3,2になっているが, ピンヘッダを代わりに使用したので, 2,3が逆転している. 

\subsection{フットプリント選定ミス}
ダイオード直後の平滑コンが, 基板のパターンに対し, でかすぎる. それゆえに, ダイオードに物理的に被っている. また, レギュレータ用の発振抑止コンが, 基板のパターンに対し小さすぎる.

\subsection{AC100Vのパターンラインが細すぎる}
AC100Vのパターンラインが, 他のところと共通になっており, 細い

\subsection{ACラインの近くにGNDベタがある}
今回, 測定の結果からすると, 影響はほとんど出ていないようだが, ACラインの付近にGNDベタを貼ると, GND揺れを起こす可能性があり, アイソレーションした方がよい.

\end{document}
